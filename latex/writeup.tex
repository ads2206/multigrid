%%%%%%%%%%%%%%%%%%%%%%%%%%%%%%%%%%%%%%%
%
%	Avi Schwarzschild and Andres Soto
%	APMA 4301: Numerical Methods for PDEs 
% Final Project: Multigrid
%
%%%%%%%%%%%%%%%%%%%%%%%%%%%%%%%%%%%%%%%

\documentclass[pdftex,12pt,a4paper]{article}
\usepackage{amsmath}
\usepackage{amssymb}		% packages that allow mathematical formatting
\usepackage{graphicx}		% package that allows you to include graphics
\usepackage{setspace}		% package that allows you to change spacing
%\onehalfspacing				% text become 1.5 spaced
\usepackage{fullpage}		% package that specifies normal margins
\addtolength{\topmargin}{-.25 in}
\usepackage{fancyhdr}
\usepackage[table]{xcolor}
\usepackage{amsmath}
\pagestyle{fancy}
\setlength{\headsep}{.5in}
\newcommand{\HRule}{\rule{\linewidth}{0.5mm}}
\graphicspath{ {images/} }
	
%-- Begin the body below ---%

\begin{document}
\lhead{Multigrid, \today}
\rhead{Schwarzschild and Soto \thepage}

\begin{titlepage}
\begin{center}


% Title


\HRule \\[0.4cm]
{ \huge \bfseries An Exploration of Multigrid Methods \\[0.4cm] }

\HRule \\[1.5cm]

\LARGE Numerical Methods for PDEs 4301\\[0.5cm]


Avi Schwarzschild and Andres Soto \\


\begin{minipage}{0.4\textwidth}
\end{minipage}

\vfill

% Bottom of the page
{\small UNIs: aks2203 and ads2206}


\end{center}
\end{titlepage}
\vspace{2 cm}

\begin{abstract}
    We used both and one and two dimenssional poisson problems to study multigrid methods for solving partial differential equations. Using iterative solvers for linear systems we show how coarsening the descritization can lead to approximations which converge to the true solution of the PDE with fewer iterations of the solver. 
\end{abstract}
%-- Sections correspond to problem numbers --%

\section{Motivation}
   
    \paragraph*{} The motivation for using coarser grids, while convergences analysis shows that finer grids should lead to more accurate approximations, comes from an observation about waves, descritizations, and aliasing. The iterative methods, often referred to as smoothers, smooth out error such that in the earlier iterations it is the high frequency components of the error that disappear. As the algorithms sweep more times the error gets smoother, containing lower frequencies and tending toward zero. The trouble with the classical iterative methods like this is that low frequencies in the error can take many iterations to smooth. In general these iterative methods are $\mathcal{O}(n^2)$. Thus, some improvement is speed is desired.

    \paragraph*{} From the Shannon Sampling Theorem we know that to retain all of the wave information, we need the descritization to have just over two points per wavelength.\footnote{C. E. Shannon, ``Communications in the presence of noise'', Proc. IRE, vol. 37, pp. 10-21, Jan. 1949.} The implication of this on our work is that the highest frequency in the error is determined by the mesh grid. Knowing this, we can coarsen the grid so that we ahve fewer smaple points of the error function and then the low frequencies with be among the higher ones still contained in the coarse-grid error. By smoothing the error on this coarser grid we can eliminate more componenets of the error than without the coarsening. We leave the explanation and implimentation of this for later sections.

\section{One Dimenssional Problem}
    
    \paragraph*{} For the one dimenssional problem we chose to demonstrate the Poisson problem with Dirichlet boundary conditions as follows:
    
    \begin{equation}
    \begin{aligned}
        \frac{d^2u}{dx^2} &= f(x) ~ , ~~ x \in (a, b), \\
        u(a) &= \alpha, \\
        u(b) &= \beta.
    \end{aligned}
    \end{equation}
    
    The values of $f(x)$, $a$, $b$, $\alpha$, and $\beta$ are specified below for two different examples.

    \subsection{Implementation}   

        \paragraph*{} 

    \subsection{Results}
        \paragraph*{}

\section{Two Dimenssional Problem}
    
    \subsection{Implementation}   
        \paragraph*{}

    \subsection{Results}
        \paragraph*{}

\section{Error Analysis}
    
    \subsection{Multigrid}   
        \paragraph*{}

    \subsection{Gauss-Seidel}
        \paragraph*{}

    \subsection{Successive Over Relaxation}
        \paragraph*{}

\section{Conclusion}













% \subparagraph{a)} 

% \subparagraph{b)} 

% $$
% \rowcolors{2}{gray!25}{white}
% \begin{tabular}{ r  l || r  l }  
% Day & Temp. & Day & Temp. \\ \hline
% 22 & 48.6544 & 91 & 32.6699 \\

% 80 & 29.9413 &  92 & 32.7286 \\

% 81 & 30.3790 & 224 & 80.9063 \\

% 82 & 30.7736 & 225 & 81.5233 \\

% 83 & 31.1268 & 226 & 82.1234 \\

% 84 & 31.4403 & 228 & 83.2654 \\

% 85 & 31.7158 & 229 & 83.8043 \\

% 86 & 31.9551 & 230 & 84.3201 \\

% 87 & 32.1600 & 231 & 84.8117 \\

% 88 & 32.3322 & 232 & 85.2780 \\

% 89 & 32.4734 & 233 & 85.7178 \\

% 90 & 32.5854 & 300 & 88.8952 \\ \hline
% \end{tabular}
% $$

% %-- Next problem: --%
% \section*{}

% \paragraph{Part 2:} 

% \subparagraph{a)} 
% $$
% \rowcolors{2}{gray!25}{white}
% \begin{tabular}{ r c c || r c c }  
% Day & Temp. & Error($\pm$) & Day & Temp. & Error($\pm$) \\ \hline
% 22 & 48.6544 & 0.0093 & 91 & 32.6699 & 22.4093 \\

% 80 & 29.9413 & 22.4093 &  92 & 32.7286 & 22.4093 \\

% 81 & 30.3790 & 22.4093 & 224 & 80.9063 & 0.1493 \\

% 82 & 30.7736 & 22.4093 & 225 & 81.5233 & 0.1493 \\

% 83 & 31.1268 & 22.4093 & 226 & 82.1234 & 0.1493 \\

% 84 & 31.4403 & 22.4093 & 228 & 83.2654 & 1.4006 \\

% 85 & 31.7158 & 22.4093 & 229 & 83.8043 & 1.4006 \\

% 86 & 31.9551 & 22.4093 & 230 & 84.3201 & 1.4006 \\

% 87 & 32.1600 & 22.4093 & 231 & 84.8117 & 1.4006 \\

% 88 & 32.3322 & 22.4093 & 232 & 85.2780 & 1.4006 \\

% 89 & 32.4734 & 22.4093 & 233 & 85.7178 & 1.4006 \\

% 90 & 32.5854 & 22.4093 & 300 & 88.8952 & 0.0093 \\ \hline
% \end{tabular}
% $$

% \section*{}

% \paragraph{Part 3:} 

% \subparagraph{a)} 

% \subparagraph{b)} hey \\ \\


% \section*{}

% \paragraph{Part 4:} 

% \subparagraph{a)} 
% 1. \\ \\

% \subparagraph{b)} Show \emph{quantitatively} why you expect your new choice of interpolation scheme to be an improvement on the original one. \\ \\
% The error estimates for the two interpolation schemes provide us with quantitative reason to think Hermite interpolation has an advantage:

% \subparagraph{c)} 

% $$
% \rowcolors{2}{gray!25}{white}
% \begin{tabular}{ r c c || r c c }  
% Day & Temp. & Error($\pm$) & Day & Temp. & Error($\pm$) \\ \hline
%    22 & 48.6545 & 0.0001 & 91 & 32.6703 & 0.0590 \\

%    80 & 29.9417 & 0.0808 & 92 & 32.7288 & 0.0115 \\

%    81 & 30.3798 & 0.2558 & 224 & 80.9063 & 0.0031 \\

%    82 & 30.7748 & 0.4465 & 225 & 81.5233 & 0.0049 \\

%    83 & 31.1283 & 0.6013 & 226 & 82.1233 & 0.0025 \\

%    84 & 31.4420 & 0.6919 & 228 & 83.2659 & 0.0086 \\

%    85 & 31.7177 & 0.7085 & 229 & 83.8053 & 0.0205 \\

%    86 & 31.9570 & 0.6563 & 230 & 84.3214 & 0.0246 \\

%    87 & 32.1617 & 0.6563 & 231 & 84.8130 & 0.0197 \\

%    88 & 32.3337 & 0.4151 & 232 & 85.2789 & 0.0103 \\

%    89 & 32.4746 & 0.2733 & 233 & 85.7182 & 0.0025 \\

%    90 & 32.5862 & 0.1488 & 300 & 88.8952 & 0.0001 \\ \hline
% \end{tabular}
% $$

% \section*{}

% \paragraph{Part 5:} 

% \subparagraph{a)} 

% \subparagraph{b)} 


% \section*{}

% \paragraph{Part 6:} 

% \subparagraph{a)} 

% \pagebreak



\end{document}