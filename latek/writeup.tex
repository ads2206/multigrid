%%%%%%%%%%%%%%%%%%%%%%%%%%%%%%%%%%%%%%%
%
%	Avi Schwarzschild and Andres Soto
%	APMA 4301: Numerical Methods for PDEs 
% Final Project: Multigrid
%
%%%%%%%%%%%%%%%%%%%%%%%%%%%%%%%%%%%%%%%

\documentclass[pdftex,12pt,a4paper]{article}
\usepackage{amsmath}
\usepackage{amssymb}		% packages that allow mathematical formatting
\usepackage{graphicx}		% package that allows you to include graphics
\usepackage{setspace}		% package that allows you to change spacing
%\onehalfspacing				% text become 1.5 spaced
\usepackage{fullpage}		% package that specifies normal margins
\addtolength{\topmargin}{-.25 in}
\usepackage{fancyhdr}
\usepackage[table]{xcolor}
\usepackage{amsmath}
\pagestyle{fancy}
\setlength{\headsep}{.5in}
\newcommand{\HRule}{\rule{\linewidth}{0.5mm}}
\graphicspath{ {images/} }
	
%-- Begin the body below ---%

\begin{document}
\lhead{Multigrid, \today}
\rhead{Schwarzschild and Soto \thepage}

\begin{titlepage}
\begin{center}


% Title


\HRule \\[0.4cm]
{ \huge \bfseries An Exploration of Multigrid Methods \\[0.4cm] }

\HRule \\[1.5cm]

\LARGE Numerical Methods for PDEs 4301\\[0.5cm]


Avi Schwarzschild and Andres Soto \\


\begin{minipage}{0.4\textwidth}
\end{minipage}

\vfill

% Bottom of the page
{\small UNIs: aks2203 and ads2206}


\end{center}
\end{titlepage}
\vspace{2 cm}
%-- Sections correspond to problem numbers --%
\section*{}

\paragraph{Part 1:} First, you are provided \emph{only} with the temperature data and the days on which the apparatus was not working. You are asked to find an estimate of the temperature on the missing days. 

\subparagraph{a)} 

\subparagraph{b)} 

$$
\rowcolors{2}{gray!25}{white}
\begin{tabular}{ r  l || r  l }  
Day & Temp. & Day & Temp. \\ \hline
22 & 48.6544 & 91 & 32.6699 \\

80 & 29.9413 &  92 & 32.7286 \\

81 & 30.3790 & 224 & 80.9063 \\

82 & 30.7736 & 225 & 81.5233 \\

83 & 31.1268 & 226 & 82.1234 \\

84 & 31.4403 & 228 & 83.2654 \\

85 & 31.7158 & 229 & 83.8043 \\

86 & 31.9551 & 230 & 84.3201 \\

87 & 32.1600 & 231 & 84.8117 \\

88 & 32.3322 & 232 & 85.2780 \\

89 & 32.4734 & 233 & 85.7178 \\

90 & 32.5854 & 300 & 88.8952 \\ \hline
\end{tabular}
$$

%-- Next problem: --%
\section*{}

\paragraph{Part 2:} 

\subparagraph{a)} 
$$
\rowcolors{2}{gray!25}{white}
\begin{tabular}{ r c c || r c c }  
Day & Temp. & Error($\pm$) & Day & Temp. & Error($\pm$) \\ \hline
22 & 48.6544 & 0.0093 & 91 & 32.6699 & 22.4093 \\

80 & 29.9413 & 22.4093 &  92 & 32.7286 & 22.4093 \\

81 & 30.3790 & 22.4093 & 224 & 80.9063 & 0.1493 \\

82 & 30.7736 & 22.4093 & 225 & 81.5233 & 0.1493 \\

83 & 31.1268 & 22.4093 & 226 & 82.1234 & 0.1493 \\

84 & 31.4403 & 22.4093 & 228 & 83.2654 & 1.4006 \\

85 & 31.7158 & 22.4093 & 229 & 83.8043 & 1.4006 \\

86 & 31.9551 & 22.4093 & 230 & 84.3201 & 1.4006 \\

87 & 32.1600 & 22.4093 & 231 & 84.8117 & 1.4006 \\

88 & 32.3322 & 22.4093 & 232 & 85.2780 & 1.4006 \\

89 & 32.4734 & 22.4093 & 233 & 85.7178 & 1.4006 \\

90 & 32.5854 & 22.4093 & 300 & 88.8952 & 0.0093 \\ \hline
\end{tabular}
$$

\section*{}

\paragraph{Part 3:} 

\subparagraph{a)} 

\subparagraph{b)} \\ \\


\section*{}

\paragraph{Part 4:} 

\subparagraph{a)} 
1. \\ \\

\subparagraph{b)} Show \emph{quantitatively} why you expect your new choice of interpolation scheme to be an improvement on the original one. \\ \\
The error estimates for the two interpolation schemes provide us with quantitative reason to think Hermite interpolation has an advantage:

\subparagraph{c)} 

$$
\rowcolors{2}{gray!25}{white}
\begin{tabular}{ r c c || r c c }  
Day & Temp. & Error($\pm$) & Day & Temp. & Error($\pm$) \\ \hline
   22 & 48.6545 & 0.0001 & 91 & 32.6703 & 0.0590 \\

   80 & 29.9417 & 0.0808 & 92 & 32.7288 & 0.0115 \\

   81 & 30.3798 & 0.2558 & 224 & 80.9063 & 0.0031 \\

   82 & 30.7748 & 0.4465 & 225 & 81.5233 & 0.0049 \\

   83 & 31.1283 & 0.6013 & 226 & 82.1233 & 0.0025 \\

   84 & 31.4420 & 0.6919 & 228 & 83.2659 & 0.0086 \\

   85 & 31.7177 & 0.7085 & 229 & 83.8053 & 0.0205 \\

   86 & 31.9570 & 0.6563 & 230 & 84.3214 & 0.0246 \\

   87 & 32.1617 & 0.6563 & 231 & 84.8130 & 0.0197 \\

   88 & 32.3337 & 0.4151 & 232 & 85.2789 & 0.0103 \\

   89 & 32.4746 & 0.2733 & 233 & 85.7182 & 0.0025 \\

   90 & 32.5862 & 0.1488 & 300 & 88.8952 & 0.0001 \\ \hline
\end{tabular}
$$

\section*{}

\paragraph{Part 5:} 

\subparagraph{a)} 

\subparagraph{b)} 


\section*{}

\paragraph{Part 6:} 

\subparagraph{a)} 

\pagebreak



\end{document}